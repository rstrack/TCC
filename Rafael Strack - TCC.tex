\documentclass[12pt]{article}

\usepackage{sbc-template}

\usepackage{graphicx,url}

\usepackage[utf8]{inputenc}  
\usepackage[brazil]{babel}  

\graphicspath{ {./img/ptbr} }

\makeatletter
\def\input@path{{./sections/ptbr}}
\makeatother
     
\sloppy

\title{Estudo Comparativo de Mecanismos de Segurança Aplicados à Autenticação e Autorização em 
Sistemas Web}

\author{Rafael Strack\inst{1}, Adriano Ferrasa\inst{1}}

\address{Departamento de Informática -- Universidade Estadual de Ponta Grossa
  (UEPG)\\
  84.030-900 -- Ponta Grossa -- PR -- Brasil
\email{rafa\_strack@hotmail.com, ferrasa@uepg.br}
}
\begin{document}

\maketitle

\begin{abstract}
  This article presents a comparative study of authentication and authorization mechanisms in web
  systems, aiming to provide an in-depth analysis of the available options and their
  characteristics. The study describes the most commonly used methods in the field, such as passwords, 
  tokens, OAuth and OpenID Connect, analyzing their advantages and disadvantages, as well as the usage 
  flow of each method. In the end, a comparison of the studied methods is conducted, evaluating their 
  effectiveness in terms of security. A proper understanding of these mechanisms is essential to ensure 
  the security of web systems and to guide the correct choice in future projects.
\end{abstract}

\begin{resumo}
  Este artigo apresenta um estudo comparativo de mecanismos de autenticação e autorização em
  sistemas web, visando fornecer uma análise aprofundada das opções disponíveis e suas
  características. O estudo descreve os métodos mais utilizados na área, como senhas, tokens, OAuth 
  e OpenID Connect, analisando suas vantagens e desvantagens, bem como o fluxo de utilização de 
  cada método. Ao final, é realizada uma comparação dos métodos estudados, avaliando-se sua eficácia 
  em termos de segurança. A compreensão adequada desses mecanismos é fundamental para garantir a 
  segurança dos sistemas web e orientar a escolha correta em projetos futuros.
\end{resumo}

\section{Introdução}

Com a expansão da internet, os sistemas web assumiram um papel crucial no cotidiano de bilhões de
pessoas em todo o mundo. Desde o uso de redes sociais até o gerenciamento de negócios online, essas
ferramentas se tornaram indispensáveis para diversas atividades. Tanto pessoas físicas quanto
empresas dependem desses sistemas para garantir a eficiência e produtividade de suas operações.
No entanto, a segurança desses sistemas é uma preocupação constante para desenvolvedores e
usuários, pois há uma série de ameaças e vulnerabilidades que podem comprometer sua integridade.

A fundação OWASP (\emph{Open Worldwide Application Security Project}) atualiza regularmente um
relatório chamado OWASP Top 10, onde são descritos os 10 riscos de segurança mais críticos em
sistemas web. Na última edição, realizada em 2021, a categoria que ficou em primeira colocação foi
a quebra de controle de acesso. Em sétima colocação, ficou a categoria de falhas de identificação e
autenticação \cite{OWASP2021}. Esses problemas são diretamente relacionados aos processos de
autenticação e autorização de usuários, os quais são essenciais para garantir a proteção adequada
dos sistemas.

De modo geral, a autenticação é o processo de validação de usuários, enquanto a autorização é o
método que fornece as permissões de acesso corretas aos recursos para usuários previamente
autenticados \cite{TUMIN2012}. Atualmente, existem diversos mecanismos de autenticação e autorização
de usuários disponíveis, como senhas, \emph{tokens}, autenticação multifator, OAuth, OpenID Connect, entre
outros. Cada um desses mecanismos apresenta características distintas, pontos positivos e negativos,
sendo fundamental garantir a correta implementação dos mecanismos escolhidos, de forma a assegurar
a efetividade da segurança dos sistemas web.

Diante desse contexto, o presente trabalho tem como objetivo realizar um estudo
comparativo dos diferentes mecanismos de autenticação e autorização, com o propósito
de fornecer uma análise aprofundada que auxilie na escolha adequada desses mecanismos
em projetos de sistemas web. O estudo visa oferecer uma compreensão ampla das
características, pontos fortes e limitações de cada mecanismo, permitindo a seleção
correta e a implementação eficiente das medidas de segurança necessárias.

\section{Autenticação e Autorização}

Na maioria dos sistemas \emph{web}, é necessário realizar um controle de acesso para que somente
certos usuários possam acessar recursos protegidos. Para isso, o mecanismo  de controle de
acesso depende de dois processos relacionados: a autenticação e a autorização
\cite{SULLIVAN2011}.

A autenticação pode ser definida como o processo de confirmação de identidade. Porém, em sistemas 
\emph{web}, devido a falta de conhecimento do mundo real, este processo pode ser complexo
\cite{CHAPMAN2012}. Existem três grupos de fatores amplamente utilizados para confirmar a
identidade de um usuário: algo que o usuário sabe, algo que o usuário é e algo que o usuário
possui. No primeiro grupo, inclui-se as senhas, PINs (\emph{Personal Identification Number}) e
frases secretas. No segundo grupo, inclui-se certificados digitais, \emph{smart cards} e
\emph{tokens} de segurança. O terceiro grupo inclui técnicas biométricas, como impressões
digitais, reconhecimento facial ou de voz, dentre outras \cite{SULLIVAN2011}.

De forma complementar, a autorização é o processo pelo qual o sistema verifica se um usuário 
previamente autenticado possui permissão para acessar um recurso ou executar uma determinada ação 
\cite{SPILCA2020}. Ela pode ser realizada de várias formas, as mais comuns baseadas em 
usuários, perfis (\emph{roles}) e através do uso do protocolo OAuth \cite{CHAPMAN2012}.

\subsection{Autenticação Básica HTTP}

A autenticação básica HTTP (\emph{Hypertext Transfer Protocol}) foi definida na especificação
HTTP/1.0 \cite{RFC1945}, porém tornou-se um padrão na RFC 2617 \cite{RFC2617}. Neste tipo de
autenticação, o servidor \emph{web} recusa uma transação caso o cliente não esteja autenticado,
desafiando-o para obter um nome de usuário e senha válidos. Este desafio de autenticação é iniciado
retornando o status HTTP 401 (não autorizado) e especificando o domínio de segurança
(\emph{security realm}) a ser acessado, com o cabeçalho \texttt{WWW-Authenticate}. Ao receber o 
desafio, o cliente abre uma caixa de diálogo para que o usuário insira as credenciais para acesso 
ao domínio. O cliente então une as informações de usuário e senha, colocando dois pontos entre 
estas, e as codifica usando o método de codificação \emph{base64} \cite{rfc4648}. Estas credenciais codificadas são 
colocadas no cabeçalho \texttt{Authorization}, e então a requisição é enviada para o servidor, que 
fará a validação das credenciais e, caso validadas, retorna-se o status HTTP 200 (OK) 
\cite{GOURLEY2002} (Figura \ref{fig:basicAuth}).

\begin{figure}[ht] 
  \centering
  \includegraphics[width=.9\textwidth]{Basic Authentication.png}
  \caption{Exemplo de autenticação básica HTTP.}
  \label{fig:basicAuth}
\end{figure}

A diretiva de domínio (\emph{realm}) utilizada nas autenticações HTTP define os espaços de proteção 
do sistema \emph{web}. Esses domínios permitem que os recursos protegidos sejam particionados, cada 
um com seu próprio esquema de autenticação e/ou autorização \cite{RFC2617}.

\subsection{Autenticação Digest}

A autenticação \emph{Digest} foi especificada na RFC 2019 \cite{RFC2019}, porém também foi 
realocada para a RFC 2617. Foi desenvolvida para ser uma alternativa mais compatível e segura  para 
a autenticação básica, corrigindo as falhas mais graves da mesma, como a falta de criptografia de 
senhas, vulnerabilidade a captura e reprodução de pacotes e proteção contra vários outros tipos comuns de ataques \cite{GOURLEY2002}.

Assim como a autenticação básica HTTP, a \emph{Digest} é baseada no paradigma 
desafio-resposta \cite{RFC7616}. A diferença é que foram adicionados diversos parâmetros nos cabeçalhos, para identificação única de desafios, nível de qualidade de proteção, especificação de algoritmo de hashing utilizado entre outros recursos \cite{CHAPMAN2012}. Por padrão, o algoritmo utilizado é o MD5, porém na RFC 7616 foram adicionados e recomendados os algoritmos SHA-256 e SHA-512/256 \cite{RFC7616}.

O parâmetro \texttt{response} é a principal parte do cabeçalho \texttt{Authorization}: ele contém 
uma concatenação criptografada de dados da requisição, como nome do usuário, \texttt{realm}, senha, 
método HTTP, URL, entre outros parâmetros, todos separados por dois pontos. \cite{CHAPMAN2012}. O cliente realiza o cálculo resultante no valor de \texttt{response}. O servidor também o faz e 
compara com o valor recebido. Caso as credenciais sejam válidas, retorna-se o status HTTP 200 (OK)
e o cabeçalho \texttt{Authentication-Info}, que contém parâmetros utilizados para uma futura 
autenticação, autenticação mútua e reenvio de parâmetros para confirmação de legitimidade. Um 
exemplo do funcionamento deste método de autenticação é mostrado na Figura \ref{fig:digestAuth}.

\begin{figure}[ht]
  \centering
  \includegraphics[width=.8\textwidth]{Digest Authentication (Simplified).png}
  \caption{Exemplo de autenticação Digest}
  \label{fig:digestAuth}
\end{figure}

Apesar da grande melhora de segurança em relação a autenticação básica, este método possui 
diversos riscos de segurança. Os cabeçalhos \texttt{WWW-Authenticate} e \texttt{Authorization} 
possuem certo nível de proteção a manipulação, porém os outros não. Ataques de repetição poderão 
ser realizados se a implementação de identificadores únicos por desafio nao for realizada. Caso não
seja estabelecida nenhuma política de força de senha, poderão ser realizados ataques de dicionário, 
tentando adivinhar a senha e outros parâmetros, visto que o nome do usuário é obtido sem esforço.
Se a requisição passar por \emph{proxies} hostis ou comprometidos, o cliente pode ficar vulnerável 
a ataques \emph{man-in-the-middle} \cite{GOURLEY2002}. Este método também é \emph{stateless}, possuindo os mesmo problema citado na autenticação básica.

\subsection{Autenticação Baseada em Sessão}

Uma sessão \emph{web} é uma troca de informações semipermanente entre um cliente e um servidor \emph{web} 
\cite{CALZAVARA2017}. O mecanismo de gerenciamento de estados para o HTTP, baseado em sessões, foi 
especificado na RFC 2109 \cite{RFC2109} com sua versão mais atual especificada na RFC 6265 
\cite{RFC6265}. Este mecanismo utiliza o termo \emph{cookie} para se referir às informações de 
estados que são passadas entre o servidor e o cliente, e salvas no cliente \cite{RFC2109}. 

A autenticação baseada em sessão é um dos métodos mais comuns de autenticação em sistemas \emph{web}. 
Neste método, após o envio de credenciais de acesso e validação do usuário por meio de uma requisição 
HTTP, o servidor gera um \emph{cookie}, armazena-o e envia-o pelo cabeçalho \texttt{Set-Cookie} da 
resposta para o cliente. O cliente salva o valor, que é enviado no cabeçalho \texttt{Cookie} em 
toda requisição para o mesmo servidor de origem \cite{PAPATHANASAKI2022} 
(Figura \ref{fig:sessionAuth}). Uma prática comum é utilizar \emph{strings} aleatórias, chamadas 
\emph{session identifiers}, para identificar a sessão. Elas precisam ser longas o suficiente para 
ser inviável adivinhá-las \cite{DRHOVA2018}.

\begin{figure}[ht]
  \centering
  \includegraphics[width=.85\textwidth]{Session-Based Authentication.png}
  \caption{Exemplo de Autenticação baseada em sessão.}
  \label{fig:sessionAuth}
\end{figure}

\

\

\subsection{Autenticação Baseada em Token}

\subsection{OAuth e OAuth 2.0}

O protocolo OAuth foi especificado na RFC 5849, fornecendo um método para clientes acessarem 
recursos de um servidor em nome de um proprietário de recurso e também um processo para que os 
usuários finais autorizem o acesso de terceiros aos seus recursos de servidor sem compartilhar suas
credenciais \cite{RFC5849}. Este método de autenticação funciona seguindo as seguintes etapas:

\begin{itemize}
\item O usuário solicita o serviço de um sistema, chamado de cliente;
\item O cliente, tendo previamente configurado acesso ao servidor de recursos, possuindo um 
identificador e um segredo compartilhado, envia uma solicitação a este servidor para receber um 
\emph{token} de solicitação;
\item O servidor valida a solicitação, enviando o \emph{token} de solicitação não-autorizado no 
corpo da resposta HTTP;
\item O cliente redireciona o agente do usuário para o servidor, que solicita o \emph{login} do 
usuário e depois a autorização para o cliente acessar o recurso;
\item O cliente recebe um \emph{token} de solicitação, que utiliza em uma nova solicitação por 
\emph{token} de acesso. Estas requisições são realizadas por meio de um canal TLS;
\item O servidor valida o \emph{token} de solicitação e envia o \emph{token} de acesso ao cliente e 
opcionalmente um \emph{token} de atualização;
\item O cliente utiliza o \emph{token} de acesso para solicitar os recursos do servidor.
\end{itemize}

Na RFC 6749 foi proposto o protocolo OAuth 2.0, tornando obsoleta sua versão anterior. Esta versão possui 
poucas semelhanças com a versão anterior, utilizando os mesmos princípios mas com fluxo diferente e 
abrangendo mais casos de uso, além de estabelecer o uso do protocolo HTTPS, possuindo todas as mensagens 
criptografadas. Por este motivo os dois protocolos não são compatíveis, coexistindo e podendo ambos serem 
suportados pelos sistemas \cite{RFC6749}. Diferente da versão 1.0, que depende de assinaturas 
criptografadas para cada requisição, a versão 2.0 utiliza \emph{tokens} portadores (\emph{bearer tokens}) 
que são protegidos devido ao uso do TLS \cite{SIRIWARDENA2014}. 

\begin{figure}[ht]
    \centering
    \includegraphics[width=.95\textwidth]{OAuth 2.0.png}
    \caption{Exemplo de autenticação utilizando OAuth 2.0}
    \label{fig:OAuth2}
\end{figure}

Um dos fluxos de funcionamento 
disponíveis, por código de autorização, é descrito pelas seguintes etapas (Figura \ref{fig:OAuth2}):

\begin{itemize}
\item O cliente (sistema solicitando acesso) direciona o proprietário do recurso para o \emph{endpoint} de autorização, no servidor
de autorização, passando o identificador do cliente, o estado local, a URI de identificação e o 
escopo solicitado;
\item O servidor de autorização autentica o proprietário do recurso e espera a concessão de acesso 
aos recursos para o cliente;
\item Após o proprietário do recurso garantir o acesso ao cliente, o servidor de autorização 
redireciona para o cliente utilizando a URI fornecida pelo cliente anteriormente, juntamente com o 
código de autorização;
\item O cliente solicita um \emph{token} de acesso ao servidor de autorização, utilizando o código
de autorização recebido anteriormente;
\item O servidor de autorização autentica o cliente, valida o código de autorização e garante que o 
URI de redirecionamento recebido corresponde ao utilizado anteriormente. Se válido, responde com um 
\emph{token} de acesso e opcionalmente com um \emph{token} de atualização;
\item O cliente utiliza o \emph{token} de acesso para solicitar os recursos do servidor \cite{RFC6749}.
\end{itemize}


\subsection{OpenID Connect}

O OpenID Connect (OIDC) é um padrão de camada de identificação em cima do protocolo OAuth 2.0. Utiliza 
todos os conceitos, \emph{tokens} e fluxos do protocolo OAuth 2.0, com a adição do fornecimento de 
atributos do usuário. Esses atributos podem ser 
fornecidos para um sistema por meio de uma API RESTful \texttt{userinfo} ou por meio de um 
\emph{token} de identificação \cite{BIEHL2019}.

O OIDC implementa autenticação como uma extensão da autorização fornecida pelo protocolo OAuth 2.0 
\cite{OIDCCORE}. Similar com o fluxo do OAuth, este método segue as seguintes etapas:

\begin{itemize}
    \item O cliente redireciona para o servidor de autenticação (Provedor OIDC) com os parâmetros 
    necessários (ID do cliente, URI de redirecionamento, etc.);
    \item O provedor apresenta uma página de login ao usuário;
    \item Após login bem sucedido, é solicitada a permissão de acesso aos recursos para o usuário;
    \item O provedor OIDC redireciona o usuário ao cliente, com o código de autorização;
    \item O cliente solicita ao provedor OIDC um \emph{token} de identificação e também geralmente
\emph{tokens} de acesso e atualização;
    \item O provedor envia ao cliente os \emph{tokens} que podem ser utilizados para acessar 
informações do usuário (\emph{endpoint} \texttt{userinfo}), recursos protegidos e solicitar novos 
\emph{tokens} \cite{OIDCCORE}.
\end{itemize}

Um exemplo de funcionamento de uma autenticação utilizando OIDC é mostrado na figura 
\ref{fig:OpenID}

\begin{figure}[ht]
    \centering
    \includegraphics[width=0.95\textwidth]{OpenID Connect.png}
    \caption{Exemplo de autenticação utilizando OpenID Connect}
    \label{fig:OpenID}
\end{figure}

\section{Materiais e Métodos}

Para o desenvolvimento do presente estudo na área de segurança da informação,  utilizou-se uma 
abordagem qualitativa. Os dados coletados relativos aos mecanismos de autenticação e autorização 
são descritivos em sua totalidade, fornecendo as características de cada um.

A coleta de dados foi realizada através de uma revisão sistemática da literatura, buscando artigos 
científicos publicados em revistas e conferências internacionais na área de segurança da informação.
Além disso, foram consultados livros e documentos técnicos, como RFCs 
(\emph{Research for Comments}), emitidos pela IETF (\emph{Internet Engineering Task Force}).

Durante a coleta de dados, foi dada ênfase aos aspectos relacionados à usabilidade, nível de 
segurança, tipos de ataques aos quais cada método é vulnerável, facilidade de implementação e 
obsolescência. Esses critérios foram escolhidos com o objetivo de fornecer uma análise comparativa 
dos métodos investigados. As informações coletadas foram registradas em uma tabela comparativa, 
permitindo análise das características de cada método e levantando quais ferramentas são mais 
indicadas para cada caso de uso.

% Realizou-se uma extensa pesquisa de dados relacionada aos métodos de autenticação HTTP 
% \emph{Basic Autentication}, HTTP \emph{Digest Authentication}, \emph{Session-Based Authentication}, 
% \emph{Token-Based Authentication}, OAuth, OAuth 2.0 e OpenID. Para a coleta de dados relacionados a 
% estes métodos, foram utilizados artigos científicos publicados em revistas e conferências 
% internacionais, livros e RFCs (\emph{Research for Comments}), documentos técnicos mantidos pela 
% IETF (\emph{Internet Enginnering Task Force}).





% \section{Resultados e Discussão}
\section{Análise preliminar}

%%%%% Basic
A Autenticação Básica HTTP é simples e de fácil implementação, porém não possui segurança. As 
credenciais do usuário podem ser facilmente decodificadas, visto que a codificação base64 é 
facilmente reversível, podendo ser realizada em poucos segundos. Também é possível realizar ataques 
de reprodução, visto que terceiros podem capturar pacotes e replicá-los, mesmo que codificados, 
podendo obter acesso ao sistema. Este tipo de autenticação não possui proteção contra \emph{proxies} 
ou \emph{middlewares}, que podem facilmente modificar o corpo da mensagem, e também são vulneráveis 
a servidores falsificados, que se passam por outros para realizar o roubo de credenciais 
\cite{GOURLEY2002}. Além destes pontos negativos a autenticação básica é \emph{stateless} (sem 
estado), cada solicitação é tratada separadamente, sem conexão contínua entre elas. Isso faz com 
que, enquanto possuir os dados, o cliente continuará mandando os cabeçalhos de autenticação em 
todas as requisições HTTP subsequentes para o mesmo domínio.

%%%%% Digest
Apesar da grande melhora de segurança em relação a autenticação básica, a autenticação 
\emph{Digest}, este método possui diversos riscos de segurança. Os cabeçalhos 
\texttt{WWW-Authenticate} e \texttt{Authorization} possuem certo nível de proteção a manipulação, 
porém os outros não. Ataques de repetição poderão ser realizados se a implementação de 
identificadores únicos por desafio nao for realizada. Caso não seja estabelecida nenhuma política de 
força de senha, poderão ser realizados ataques de dicionário, tentando adivinhar a senha e outros 
parâmetros, visto que o nome do usuário é obtido sem esforço. Se a requisição passar por 
\emph{proxies} hostis ou comprometidos, o cliente pode ficar vulnerável a ataques 
\emph{man-in-the-middle} \cite{GOURLEY2002}. Este método também é \emph{stateless}, possuindo o 
mesmo problema citado na autenticação básica.

%%%%% Session
Em relação à autenticação baseada em sessão, a grande vantagem é a diminuição do envio das 
credenciais do usuário nas requisições, diminuindo a janela de ataques, já que os \emph{cookies} 
são utilizados para a validação das requisições. Por outro lado, os cookies podem ser lidos por 
outros aplicativos, tornando o sistema exposto a ataques \emph{Cross Site Scripting} (XSS) e 
\emph{Cross Site Request Forgery} (CSRF). Para evitar ataques XSS, pode-se definir no \emph{cookie} 
a \emph{flag} \texttt{http-only}, que faz com que o acesso por APIs do lado cliente seja negado 
\cite{PAPATHANASAKI2022}.

%%%%% OAuth
O protocolo OAuth quando foi publicado, em 2007, tornou-se rapidamente o padrão na indústria para 
delegação de acesso na web. Porém, teve problemas no domínio empresarial, devido ao seu desempenho. 
A comunidade percebeu que o protocolo não era escalável: exige gerenciamento de estado em diferentes 
etapas, gerenciamento de credenciais temporárias e não fornece isolamento do servidor de autorização 
do próprio servidor de recursos protegidos \cite{NOUREDDINE2011}. O OAuth 2.0 resolveu estes 
problemas, facilitando o fluxo ao substituir as assinaturas por \emph{bearer tokens}, utilizando TLS 
durante todo o fluxo, não somente no \emph{handshake} inicial e definindo o servidor de autorização 
separadamente do servidor de recurso, que traz maior flexibilidade \cite{SIRIWARDENA2014}.

Em relação a ataques, ambos são suscetíveis a CSRF. A versão 1.0 é mais suscetível devido a 
falta de uso de TLS em todo seu fluxo, enquanto a versão 2.0 é suscetível a este ataque caso as 
diretrizes de implementação não forem seguidas \cite{FETT2016}. Também acaba se tornando suscetível 
a ataques de \emph{pishing} e \emph{spoofing}, caso não forem seguidas as diretrizes \cite{RFC6819}



\begin{table}[!ht]
    \centering
    \caption{Tabela comparativa entre aspectos de ferramentas de autenticação e autorização.}
    \label{tab:comparativeTable}
    \begin{tabular}{|l|l|l|l|}
    % Cabeçalho
    \hline
    Mecanismo
    % & Usabilidade
    & \begin{tabular}[c]{@{}l@{}}Nível de \\ Segurança\end{tabular}
    & Vulnerabilidades
    % & \begin{tabular}[c]{@{}l@{}}Facilidade de \\ Implementação\end{tabular}
    & Obsolescência
    \\ \hline
    
    % Basic Auth
    \begin{tabular}[c]{@{}l@{}}HTTP Basic \\ Authentication\end{tabular}
    % & Baixa
    & \begin{tabular}[c]{@{}l@{}}Muito\\baixo\end{tabular}   
    & \begin{tabular}[c]{@{}l@{}}
        Força-Bruta,\\
        Repetição, CSRF\\
        \emph{Man-in-the-middle}\\
        \emph{Sniffing}, \emph{Spoofing}
    \end{tabular}
    % &
    & \begin{tabular}[c]{@{}l@{}}Não recomendado\\para sistemas\\modernos\\\end{tabular}
    \\ \hline
    
    % Digest Auth
    \begin{tabular}[c]{@{}l@{}}HTTP Digest \\ Authentication\end{tabular}
    % & Baixa
    & Baixo
    & \begin{tabular}[c]{@{}l@{}}
        Força-Bruta\\
        Repetição, CSRF\\
        \emph{Man-in-the-middle}\\
        \emph{Sniffing}, \emph{Spoofing}
    \end{tabular}
    % &
    & \begin{tabular}[c]{@{}l@{}}Não recomendado\\para sistemas\\modernos\\\end{tabular}
    \\ \hline
    
    % Session-Based Auth
    \begin{tabular}[c]{@{}l@{}}Session-Based \\ Authentication\end{tabular}
    % & Alta
    & Médio
    & \begin{tabular}[c]{@{}l@{}}
        Força-Bruta, CSRF,\\
        XSS (sem HTTPS), \\
        Fixação de sessão,\\
        \emph{Sniffing}
    \end{tabular}
    % &
    & \begin{tabular}[c]{@{}l@{}}Amplamente utilizado\\ mas possui opções\\ modernas disponíveis\\\end{tabular}\\ \hline
    
    % Token-Based Auth
    \begin{tabular}[c]{@{}l@{}}Token-Based \\ Authentication\end{tabular}
    % & Alta
    & Alto
    & \begin{tabular}[c]{@{}l@{}}
        Força-Bruta, CSRF\\
        XSS, Remoção de \\
        assinatura
    \end{tabular}
    % &
    & \begin{tabular}[c]{@{}l@{}}Amplamente utilizado\\ mas possui opções\\ modernas disponíveis\\\end{tabular}\\ \hline
    
    % OAuth 1.0 e 2.0
    \begin{tabular}[c]{@{}l@{}}OAuth e\\OAuth 2.0\end{tabular}
    % & Alta
    & \begin{tabular}[c]{@{}l@{}}Muito\\Alto\end{tabular}
    & \begin{tabular}[c]{@{}l@{}}
        CSRF, Repetição, \\
        Redirecionamento,\\
        \emph{spoofing}
    \end{tabular}
    % &
    & \begin{tabular}[c]{@{}l@{}}OAuth: Obsoleto\\porém ainda aplicável\\OAuth 2.0: Amplamente\\utilizado e recomendado\\\end{tabular}
    \\ \hline
    
    % OpenID Connect
    OpenID Connect
    % & Alta
    & \begin{tabular}[c]{@{}l@{}}Muito\\Alto\end{tabular}
    & \begin{tabular}[c]{@{}l@{}}
        CSRF, Injeção, \\
        \emph{spoofing}
    \end{tabular}
    % &
    & \begin{tabular}[c]{@{}l@{}}Amplamente\\utilizado e\\recomendado\\\end{tabular}
    \\ \hline
    \end{tabular}
    \end{table}

\section{Conclusão}

Este trabalho teve como objetivo realizar um estudo comparativo dos diferentes mecanismos de 
autenticação e autorização em sistemas web, visando fornecer uma análise aprofundada que auxilie na 
escolha adequada desses mecanismos em projetos. O estudo buscou oferecer uma compreensão ampla das 
características, pontos fortes e limitações de cada mecanismo, a fim de permitir a seleção correta 
e a implementação eficiente das medidas de segurança necessárias.

Ao longo do trabalho, foram apresentados mecanismos de autenticação e autorização, dentre eles a
autenticação básica HTTP, autenticação \emph{digest}, autenticações baseadas em sessão e em 
\emph{tokens}, OAuth, OAuth 2.0 e OpenID Connect. Foram também apresentadas algumas 
características relacionadas a usabilidade, nível de segurança, vulnerabilidades, implementação e 
obsolescência destes mecanismos. Ao fim, apresentou-se uma tabela comparativa entre os métodos, para
sumarizar as características de cada um, facilitando a análise dos dados. 

A segurança dos sistemas web é um aspecto crucial que não pode ser negligenciado. A escolha e 
implementação correta dos mecanismos de autenticação e autorização são fundamentais para garantir 
a proteção dos dados e recursos dos usuários. Por meio deste estudo comparativo, espera-se que os 
desenvolvedores e profissionais da área possam tomar decisões mais assertivas ao selecionar os 
mecanismos de acordo com as necessidades de cada projeto, contribuindo assim para a 
construção de sistemas web mais seguros e confiáveis.

\bibliographystyle{sbc}
\bibliography{references}

\end{document}
