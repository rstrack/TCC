\section{Resultados e Discussão}

Diante dos métodos estudados e avaliados, as características relativas a nível de segurança, 
implementação, vulnerabilidades e obsolescência são exploradas neste item e sumarizadas (Tabela 
\ref{tab:comparativeTable}).

%%%%% Basic
A autenticação básica HTTP é simples e de fácil implementação, porém não possui mecanismos de 
garantia efetiva de segurança. As credenciais do usuário podem ser facilmente decodificadas, visto 
que a codificação base64 é facilmente reversível, podendo ser realizada em poucos segundos. Também 
é possível realizar ataques de repetição, visto que terceiros podem capturar pacotes e replicá-los, 
mesmo que codificados, podendo obter acesso ao sistema. Este tipo de autenticação não possui proteção 
contra \emph{proxies} ou \emph{middlewares}, que podem facilmente modificar o corpo da mensagem, e 
também são vulneráveis a servidores falsificados, que se passam por outros para realizar o roubo de 
credenciais \cite{GOURLEY2002}.

Além destes pontos negativos, a autenticação básica não possui desconexão, pois o navegador mantém 
os dados em \emph{cache} por padrão, até ser fechado. A documentação deste método cita a sua falta 
de segurança, indicando seu uso somente para propósitos simples de identificação, sem risco de 
acesso a informações sensíveis ou valiosas \cite{RFC7617}. 

%%%%% Digest
Apesar do incremento de segurança em relação a autenticação básica, a autenticação 
\emph{Digest} ainda possui diversos riscos de segurança. Os cabeçalhos 
\texttt{WWW-Authenticate} e \texttt{Authorization} possuem certo nível de proteção a manipulação, 
porém todos os outros cabeçalhos não possuem. Ataques de repetição podem ser realizados se a 
implementação de identificadores únicos por desafio não for realizada. Caso não seja estabelecida 
nenhuma política de força de senha, podem ser realizados ataques de dicionário, tentando adivinhar 
a senha e outros parâmetros, visto que o nome do usuário é obtido sem esforço. Se a requisição 
passar por \emph{proxies} hostis ou comprometidos, o cliente pode ficar vulnerável a ataques 
\emph{man-in-the-middle} \cite{GOURLEY2002}. 

Assim como a autenticação básica, a Digest não possui desconexão, possuindo o mesmo problema citado
anteriormente. Sua documentação cita que, pelos padrões modernos de criptografia, este método 
é fraco, porém um substituto muito superior a autenticação básica \cite{RFC7616}.

%%%%% Session
Em relação à autenticação baseada em sessão, a grande vantagem é a diminuição do envio das 
credenciais do usuário nas requisições, diminuindo a janela de ataques, já que os \emph{cookies} 
são utilizados para a validação das requisições. Por outro lado, os \emph{cookies} podem ser lidos por 
outros aplicativos, tornando o sistema exposto a ataques \emph{Cross Site Scripting} (XSS) e 
\emph{Cross Site Request Forgery} (CSRF). Para evitar ataques XSS, pode-se definir no \emph{cookie} 
a \emph{flag} \texttt{http-only}, que faz com que o acesso por APIs do lado cliente seja negado 
\cite{PAPATHANASAKI2022}. 

Existe a possibilidade do ataque de fixação de sessão, onde o atacante insere seu próprio valor
dentro do \emph{cookie}, esperando o usuário entrar no sistema com esse identificador. Se o servidor
aceitá-lo, o atacante obtém acesso ao servidor, podendo obter informações privadas sobre a vítima
e realizar ações usando sua identidade \cite{DRHOVA2018}.

Se não forem transportados por um canal seguro, como TLS, as informações nos cabeçalhos referentes 
aos \emph{cookies} podem ser visualizadas. Por este motivo, além de enviados por um canal seguro, 
os \emph{cookies} devem ser criptografados e assinados pelo servidor\cite{RFC6265}.

A autenticação baseada em sessão é \emph{stateful}, contendo todos os identificadores de sessão do
lado do servidor, causando uma sobrecarga no mesmo e dificultando o gerenciamento dessas sessões em
múltiplos servidores, diferentemente das outras técnicas abordadas, que são \emph{stateless} 
\cite{BALAJ2017}.

%%%%% Token

Sobre a autenticação baseada em \emph{tokens}, mais especificamente utilizando JWT, que é a mais 
comum, pode-se listar algumas vulnerabilidades caso boas práticas não sejam aplicadas. Caso seja
validado pelo servidor o uso de JWTs sem assinatura, pode-se sofrer ataque de remoção de assinatura,
removendo a terceira parte do \emph{token} e alterando seu cabeçalho. Caso os \emph{tokens} sejam
armazenados em \emph{cookies}, é possível a realização de ataques CSRF, além de ataques XSS, caso a
\emph{flag} \texttt{http-only} não seja declarada \cite{PEYROTT2018}. Caso a chave de assinatura
do JWT possua baixa entropia, um ataque de força-bruta pode ser realizado para descobri-la 
\cite{RFC7515}.

%%%%% OAuth
O protocolo OAuth quando foi publicado, em 2007, tornou-se rapidamente o padrão na indústria para 
delegação de acesso na \emph{web}. Porém, teve problemas no domínio empresarial, devido ao seu desempenho. 
A comunidade percebeu que o protocolo não era escalável: exige gerenciamento de estado em diferentes 
etapas, gerenciamento de credenciais temporárias e não fornece isolamento do servidor de autorização 
do próprio servidor de recursos protegidos \cite{NOUREDDINE2011}. O OAuth 2.0 resolveu estes 
problemas, facilitando o fluxo ao substituir as assinaturas por \emph{bearer tokens}, utilizando TLS 
durante todo o fluxo, não somente no \emph{handshake} inicial e definindo o servidor de autorização 
separadamente do servidor de recurso, que traz maior flexibilidade \cite{SIRIWARDENA2014}.

Em relação a ataques, ambos são suscetíveis a CSRF. A versão 1.0 é mais suscetível devido a 
falta de uso de TLS em todo seu fluxo, enquanto a versão 2.0 é suscetível a este ataque caso as 
diretrizes de implementação não forem seguidas \cite{FETT2016}. Também acaba se tornando suscetível 
a ataques de \emph{phishing} e \emph{spoofing}, caso não forem seguidas as diretrizes \cite{RFC6819}

%%%%% OIDC

O OpenID Connect, por ser uma camada sobre o protocolo OAuth 2.0, possui muitas vulnerabilidades 
em comum com este. Porém, o OIDC possui algumas em particular que são aplicadas somente a
ele: ataques de falsificação de solicitações no servidor, que são facilitados pela descoberta 
\emph{ad hoc} e pelos recursos de registro dinâmico; ataques de injeção, permitidos devido aos mesmos
recursos; ataques CSRF, devido ao recurso de inicialização de login de terceiros e ataques ao
/emph{token} de identificação, que não existe no OAuth \cite{FETT2017}.

Da mesma forma, algumas das vulnerabilidades do OAuth 2.0 não são aplicadas ao OIDC: 
suas configurações são menos propensas a ataques de redirecionamento aberto, uma vez que 
\emph{placeholders} não são permitidos em URIs de redirecionamento; o TLS é obrigatório para 
algumas mensagens no OIDC, enquanto é opcional no OAuth 2.0; o valor \texttt{nonce} pode evitar 
alguns ataques de repetição quando o valor do estado não é usado ou vaza para um invasor 
\cite{FETT2017}.

\begin{table}[!ht]
    \centering
    \caption{Tabela comparativa entre aspectos de ferramentas de autenticação e autorização.}
    \label{tab:comparativeTable}
    \begin{tabular}{|l|l|l|l|}
    % Cabeçalho
    \hline
    Mecanismo
    % & Usabilidade
    & \begin{tabular}[c]{@{}l@{}}Nível de \\ Segurança\end{tabular}
    & Vulnerabilidades
    % & \begin{tabular}[c]{@{}l@{}}Facilidade de \\ Implementação\end{tabular}
    & Obsolescência
    \\ \hline
    
    % Basic Auth
    \begin{tabular}[c]{@{}l@{}}HTTP Basic \\ Authentication\end{tabular}
    % & Baixa
    & \begin{tabular}[c]{@{}l@{}}Muito\\baixo\end{tabular}   
    & \begin{tabular}[c]{@{}l@{}}
        Força-Bruta,\\
        Repetição, CSRF\\
        \emph{Man-in-the-middle}\\
        \emph{Sniffing}, \emph{Spoofing}
    \end{tabular}
    % &
    & \begin{tabular}[c]{@{}l@{}}Não recomendado\\para sistemas\\modernos\\\end{tabular}
    \\ \hline
    
    % Digest Auth
    \begin{tabular}[c]{@{}l@{}}HTTP Digest \\ Authentication\end{tabular}
    % & Baixa
    & Baixo
    & \begin{tabular}[c]{@{}l@{}}
        Força-Bruta\\
        Repetição, CSRF\\
        \emph{Man-in-the-middle}\\
        \emph{Sniffing}, \emph{Spoofing}
    \end{tabular}
    % &
    & \begin{tabular}[c]{@{}l@{}}Não recomendado\\para sistemas\\modernos\\\end{tabular}
    \\ \hline
    
    % Session-Based Auth
    \begin{tabular}[c]{@{}l@{}}Session-Based \\ Authentication\end{tabular}
    % & Alta
    & Médio
    & \begin{tabular}[c]{@{}l@{}}
        Força-Bruta, CSRF,\\
        XSS (sem HTTPS), \\
        Fixação de sessão,\\
        \emph{Sniffing}
    \end{tabular}
    % &
    & \begin{tabular}[c]{@{}l@{}}Amplamente utilizado\\ mas possui opções\\ modernas disponíveis\\\end{tabular}\\ \hline
    
    % Token-Based Auth
    \begin{tabular}[c]{@{}l@{}}Token-Based \\ Authentication\end{tabular}
    % & Alta
    & Alto
    & \begin{tabular}[c]{@{}l@{}}
        Força-Bruta, CSRF\\
        XSS, Remoção de \\
        assinatura
    \end{tabular}
    % &
    & \begin{tabular}[c]{@{}l@{}}Amplamente utilizado\\ mas possui opções\\ modernas disponíveis\\\end{tabular}\\ \hline
    
    % OAuth 1.0 e 2.0
    \begin{tabular}[c]{@{}l@{}}OAuth e\\OAuth 2.0\end{tabular}
    % & Alta
    & \begin{tabular}[c]{@{}l@{}}Muito\\Alto\end{tabular}
    & \begin{tabular}[c]{@{}l@{}}
        CSRF, Repetição, \\
        Redirecionamento,\\
        \emph{spoofing}
    \end{tabular}
    % &
    & \begin{tabular}[c]{@{}l@{}}OAuth: Obsoleto\\porém ainda aplicável\\OAuth 2.0: Amplamente\\utilizado e recomendado\\\end{tabular}
    \\ \hline
    
    % OpenID Connect
    OpenID Connect
    % & Alta
    & \begin{tabular}[c]{@{}l@{}}Muito\\Alto\end{tabular}
    & \begin{tabular}[c]{@{}l@{}}
        CSRF, Injeção, \\
        \emph{spoofing}
    \end{tabular}
    % &
    & \begin{tabular}[c]{@{}l@{}}Amplamente\\utilizado e\\recomendado\\\end{tabular}
    \\ \hline
    \end{tabular}
    \end{table}

% A Tabela \ref{tab:comparativeTable} sumariza os dados levantados. O nível de segurança implica das
% vulnerabilidades apresentadas sobre cada método com base na revisão bibliográfica realizada, assim 
% como a obsolescência. 
Dentre os métodos analisados, em relação a baixa segurança, as autenticações HTTP básica e Digest são 
recomendadas somente a propósito de identificação \cite{RFC2617}. Caso o uso seja para acesso a 
conteúdo privado, indica-se a utilização de um método mais recente e com maior segurança. Ainda, em
relação a recursos de segurança aprimorados, o método que demonstra ser mais completo é o OpenID
Connect: utiliza como base o OAuth 2.0, um padrão seguro e amplamente utilizado 
de autorização, com adição de uma camada de autenticação, além de melhoras em relação a definições 
e segurança, dada a obrigatoriedade do uso de SSL. Apesar disso, deve-se levar em consideração a 
complexidade de implementação e a necessidade de cada caso de uso, tornando a escolha do mecanismo
a ser utilizado uma decisão dos desenvolvedores. Independente de qual a escolha, deve-se seguir as
diretrizes impostas na especificação de cada mecanismo, para uma menor vulnerabilidade.