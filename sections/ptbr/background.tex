\section{Autenticação e Autorização}

Na maioria dos sistemas \emph{web}, é necessário realizar um controle de acesso para que somente
certos usuários possam acessar recursos protegidos. Para isso, o mecanismo  de controle de
acesso depende de dois processos relacionados: a autenticação e a autorização
\cite{SULLIVAN2011}.

A autenticação pode ser definida como o processo de confirmação de identidade. Porém, em sistemas 
\emph{web}, devido a falta de conhecimento do mundo real, este processo pode não ser simples
\cite{CHAPMAN2012}. Existem três grupos de fatores amplamente utilizados para confirmar a
identidade de um usuário: algo que o usuário sabe, algo que o usuário é e algo que o usuário
possui. No primeiro grupo, inclui-se as senhas, PINs (\emph{Personal Identification Number}) e
frases secretas. No segundo grupo, inclui-se certificados digitais, \emph{smart cards} e
\emph{tokens} de segurança. O terceiro grupo inclui técnicas biométricas, como como impressões
digitais, reconhecimento facial ou de voz, entre outras \cite{SULLIVAN2011}.

De forma complementar, a autorização é o processo pelo qual o sistema verifica se um usuário 
previamente autenticado possui permissão para acessar um recurso ou executar uma determinada ação 
\cite{SPILCA2020}. Ela pode ser realizada de várias formas, porém as mais comuns são as baseadas em 
usuários, perfis (\emph{roles}) e com o uso do protocolo OAuth \cite{CHAPMAN2012}.