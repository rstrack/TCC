\section{Materiais e Métodos}

Para o desenvolvimento do presente estudo na área de segurança da informação, utilizou-se uma 
abordagem qualitativa. Os dados coletados relativos aos mecanismos de autenticação e autorização 
são descritivos em sua totalidade, fornecendo as características de cada um.

A coleta de dados foi realizada através de uma revisão sistemática da literatura, buscando artigos 
científicos publicados em revistas e conferências internacionais na área de segurança da informação.
Além disso, foram consultados livros e documentos técnicos, como RFCs 
(\emph{Request for Comments}), emitidos pela IETF (\emph{Internet Engineering Task Force}).

Durante a coleta de dados, foi dada ênfase aos aspectos relacionados à usabilidade, nível de 
segurança, vulnerabilidades, implementação e obsolescência, os quais representam características 
fundamentais que visam direcionar a aplicação correta dos métodos apresentados. As informações 
coletadas foram registradas em um quadro comparativo, utilizando como ponto principal as 
vulnerabilidades de cada método estudado, Esta característica acaba por implicar em 
outros pontos importantes a se considerar na escolha de um método para implementar em 
um sistema \emph{web}, como a complexidade e o nível de segurança. Assim, permite-se a 
análise das características de cada método, possibilitando ao desenvolvedor escolher de forma mais 
assertiva quais ferramentas são mais indicadas para cada caso de uso.



