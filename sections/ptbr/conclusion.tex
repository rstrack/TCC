\section{Conclusão}

Este trabalho apresentou um estudo comparativo dos diferentes mecanismos de 
autenticação e autorização em sistemas \emph{web}, visando fornecer uma análise aprofundada que auxilie na 
escolha adequada desses mecanismos em projetos. O estudo buscou oferecer uma compreensão ampla das 
características, pontos fortes e limitações de cada mecanismo, a fim de permitir a seleção correta 
e a implementação eficiente das medidas de segurança necessárias.

Ao longo do trabalho, foram apresentados mecanismos de autenticação e autorização, dentre eles a
autenticação básica HTTP, autenticação \emph{digest}, autenticações baseadas em sessão e em 
\emph{tokens}, OAuth, OAuth 2.0 e OpenID Connect. Foram também apresentadas algumas 
características relacionadas a usabilidade, nível de segurança, vulnerabilidades, implementação e 
obsolescência destes mecanismos. Ao fim, apresentou-se um quadro comparativo entre os métodos, para
sumarizar as características de cada um, facilitando a análise dos dados. 

A segurança dos sistemas \emph{web} é um aspecto crucial que não pode ser negligenciado. A escolha e 
implementação correta dos mecanismos de autenticação e autorização são fundamentais para garantir 
a proteção dos dados e recursos dos usuários. Por meio deste estudo comparativo, espera-se que os 
desenvolvedores e profissionais da área possam tomar decisões mais assertivas ao selecionar os 
mecanismos de acordo com as necessidades de cada projeto, contribuindo assim para a 
construção de sistemas \emph{web} mais seguros e confiáveis.