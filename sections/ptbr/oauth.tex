\subsection{OAuth e OAuth 2.0}

O protocolo OAuth foi especificado na RFC 5849, fornecendo um método para clientes acessarem 
recursos de um servidor em nome de um proprietário de recurso e também um processo para que os 
usuários finais autorizem o acesso de terceiros aos seus recursos de servidor sem compartilhar suas
credenciais \cite{RFC5849}. Este método de autenticação funciona seguindo as seguintes etapas:

\begin{itemize}
\item O usuário solicita o serviço de um sistema, chamado de cliente;
\item O cliente, tendo previamente configurado acesso ao servidor de recursos, possuindo um 
identificador e um segredo compartilhado, envia uma solicitação a este servidor para receber um 
\emph{token} de solicitação;
\item O servidor valida a solicitação, enviando o \emph{token} de solicitação não-autorizado no 
corpo da resposta HTTP;
\item O cliente redireciona o agente do usuário para o servidor, que solicita o \emph{login} do 
usuário e depois a autorização para o cliente acessar o recurso;
\item O cliente recebe um \emph{token} de solicitação, que utiliza em uma nova solicitação por 
\emph{token} de acesso. Estas requisições são realizadas por meio de um canal TLS;
\item O servidor valida o \emph{token} de solicitação e envia o \emph{token} de acesso ao cliente e 
opcionalmente um \emph{token} de atualização;
\item O cliente utiliza o \emph{token} de acesso para solicitar os recursos do servidor.
\end{itemize}

Na RFC 6749 foi proposto o protocolo OAuth 2.0, tornando obsoleta sua versão anterior. Esta versão possui 
poucas semelhanças com a versão anterior, utilizando os mesmos princípios mas com fluxo diferente e 
abrangendo mais casos de uso, além de estabelecer o uso do protocolo HTTPS, possuindo todas as mensagens 
criptografadas. Por este motivo os dois protocolos não são compatíveis, coexistindo e podendo ambos serem 
suportados pelos sistemas \cite{RFC6749}. Diferente da versão 1.0, que depende de assinaturas 
criptografadas para cada requisição, a versão 2.0 utiliza \emph{tokens} portadores (\emph{bearer tokens}) 
que são protegidos devido ao uso do TLS \cite{SIRIWARDENA2014}. 

\begin{figure}[ht]
    \centering
    \includegraphics[width=.95\textwidth]{OAuth 2.0.png}
    \caption{Exemplo de autenticação utilizando OAuth 2.0}
    \label{fig:OAuth2}
\end{figure}

Um dos fluxos de funcionamento 
disponíveis, por código de autorização, é descrito pelas seguintes etapas (Figura \ref{fig:OAuth2}):

\begin{itemize}
\item O cliente (sistema solicitando acesso) direciona o proprietário do recurso para o \emph{endpoint} de autorização, no servidor
de autorização, passando o identificador do cliente, o estado local, a URI de identificação e o 
escopo solicitado;
\item O servidor de autorização autentica o proprietário do recurso e espera a concessão de acesso 
aos recursos para o cliente;
\item Após o proprietário do recurso garantir o acesso ao cliente, o servidor de autorização 
redireciona para o cliente utilizando a URI fornecida pelo cliente anteriormente, juntamente com o 
código de autorização;
\item O cliente solicita um \emph{token} de acesso ao servidor de autorização, utilizando o código
de autorização recebido anteriormente;
\item O servidor de autorização autentica o cliente, valida o código de autorização e garante que o 
URI de redirecionamento recebido corresponde ao utilizado anteriormente. Se válido, responde com um 
\emph{token} de acesso e opcionalmente com um \emph{token} de atualização;
\item O cliente utiliza o \emph{token} de acesso para solicitar os recursos do servidor \cite{RFC6749}.
\end{itemize}
