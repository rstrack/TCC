\subsection{Autenticação Básica HTTP}

A autenticação básica HTTP (\emph{Hypertext Transfer Protocol}) foi definida na especificação
HTTP/1.0 \cite{RFC1945}, porém tornou-se um padrão na RFC 2617 \cite{RFC2617}. Neste tipo de
autenticação, o servidor \emph{web} recusa uma transação caso o cliente não esteja autenticado,
desafiando-o para obter um nome de usuário e senha válidos. Este desafio de autenticação é iniciado
retornando o status HTTP 401 (não autorizado) e especificando o domínio de segurança
(\emph{security realm}) a ser acessado, com o cabeçalho \texttt{WWW-Authenticate}. Ao receber o 
desafio, o cliente abre uma caixa de diálogo para que o usuário insira as credenciais para acesso 
ao domínio. O cliente então une as informações de usuário e senha, colocando dois pontos entre 
estas, e as codifica usando o método de codificação \emph{base64} \cite{rfc4648}. Estas credenciais codificadas são 
colocadas no cabeçalho \texttt{Authorization}, e então a requisição é enviada para o servidor, que 
fará a validação das credenciais e, caso validadas, retorna-se o status HTTP 200 (OK) 
\cite{GOURLEY2002} (Figura \ref{fig:basicAuth}).

\begin{figure}[ht] 
  \centering
  \includegraphics[width=.9\textwidth]{Basic Authentication.png}
  \caption{Exemplo de autenticação básica HTTP.}
  \label{fig:basicAuth}
\end{figure}

A diretiva de domínio (\emph{realm}) utilizada nas autenticações HTTP define os espaços de proteção 
do sistema \emph{web}. Esses domínios permitem que os recursos protegidos sejam particionados, cada 
um com seu próprio esquema de autenticação e/ou autorização \cite{RFC2617}.