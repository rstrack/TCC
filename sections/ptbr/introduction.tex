\section{Introdução}

Com a expansão da internet, os sistemas web assumiram um papel crucial no cotidiano de bilhões de
pessoas em todo o mundo. Desde o uso de redes sociais até o gerenciamento de negócios online, essas
ferramentas se tornaram indispensáveis para diversas atividades. Tanto pessoas físicas quanto
empresas dependem desses sistemas para garantir a eficiência e produtividade de suas operações.
No entanto, a segurança desses sistemas é uma preocupação constante para desenvolvedores e
usuários, pois há uma série de ameaças e vulnerabilidades que podem comprometer sua integridade.

A fundação OWASP (\emph{Open Worldwide Application Security Project}) atualiza regularmente um
relatório chamado OWASP Top 10, onde são descritos os 10 riscos de segurança mais críticos em
sistemas web. Na última edição, realizada em 2021, a categoria que ficou em primeira colocação foi
a quebra de controle de acesso. Em sétima colocação, ficou a categoria de falhas de identificação e
autenticação \cite{OWASP2021}. Esses problemas são diretamente relacionados aos processos de
autenticação e autorização de usuários, os quais são essenciais para garantir a proteção adequada
dos sistemas.

De modo geral, a autenticação é o processo de validação de usuários, enquanto a autorização é o
método que fornece as permissões de acesso corretas aos recursos para usuários previamente
autenticados \cite{TUMIN2012}. Atualmente, existem diversos mecanismos de autenticação e autorização
de usuários disponíveis, como senhas, \emph{tokens}, autenticação multifator, OAuth, OpenID Connect, entre
outros. Cada um desses mecanismos apresenta características distintas, pontos positivos e negativos,
sendo fundamental garantir a correta implementação dos mecanismos escolhidos, de forma a assegurar
a efetividade da segurança dos sistemas web.

Diante desse contexto, o presente trabalho tem como objetivo realizar um estudo
comparativo dos diferentes mecanismos de autenticação e autorização, com o propósito
de fornecer uma análise aprofundada que auxilie na escolha adequada desses mecanismos
em projetos de sistemas web. O estudo visa oferecer uma compreensão ampla das
características, pontos fortes e limitações de cada mecanismo, permitindo a seleção
correta e a implementação eficiente das medidas de segurança necessárias.