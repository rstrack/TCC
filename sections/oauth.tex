\subsection{OAuth e Oauth 2.0}

O protocolo OAuth foi especificado na RFC 5849, fornecendo um método para clientes acessarem recursos de um servidor em nome de um proprietário de recurso e também um processo para que os usuários finais autorizem o acesso de terceiros aos seus recursos de servidor sem compartilhar suas
credenciais \cite{RFC5849}. Este método de autenticação funciona seguindo as seguintes etapas:

\begin{itemize}
\item O usuário solicita o serviço de um sistema, chamado de cliente;
\item O cliente, tendo previamente configurado acesso ao servidor de recursos, possuindo um identificador e um segredo compartilhado, envia uma solicitação a este servidor para receber um \emph{token} de solicitação;
\item O servidor valida a solicitação, enviando o \emph{token} de solicitação não-autorizado no corpo da resposta HTTP;
\item O cliente redireciona o agente do usuário para o servidor, que solicita o \emph{login} do usuário e depois a autorização para o cliente acessar o recurso;
\item O cliente recebe um \emph{token} de solicitação, que utiliza em uma nova solicitação por \emph{token} de acesso. Estas requisições são realizadas por meio de um canal TLS;
\item O servidor valida o \emph{token} de solicitação e envia o \emph{token} de acesso ao cliente e opcionalmente um \emph{token} de atualização;
\item O cliente utiliza o \emph{token} de acesso para solicitar os recursos do servidor.
\end{itemize}

Na RFC 6749 foi especificado o OAuth 2.0, tornando obsoleta sua versão anterior. Esta versão possui poucas semelhanças com a versão anterior, utilizando os mesmos princípios mas com fluxo diferente e abrangendo mais casos de uso. Por este motivo as duas não são compatíveis, coexistindo e podendo ambas serem suportadas pelos sistemas \cite{RFC6749}. Seu funcionamento é dado pelas seguintes etapas:

\begin{itemize}
\item O cliente solicita permissão de acesso ao usuário;
\item O cliente recebe a permissão e solicita um \emph{token} de acesso ao servidor de autenticação;
\item O servidor de autenticação autentica o cliente e valida suas permissões, enviando a ele o \emph{token} de acesso e opcionalmente um \emph{token} de atualização;
\item O cliente solicita o recurso protegido ao servidor de recursos, que valida o \emph{token} de acesso e atende a solicitação.
\end{itemize}

Um exemplo de funcionamento de uma autenticação utilizando OAuth 2.0 é mostrado na figura \ref{fig:OAuth2}

\begin{figure}[ht]
    \centering
    \includegraphics[width=.8\textwidth]{OAuth 2.0.png}
    \caption{Exemplo de autenticação utilizando OAuth 2.0}
    \label{fig:OAuth2}
\end{figure}