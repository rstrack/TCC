\section{Resultados e Discussão}

%%%%%% Basic
% A Autenticação Básica HTTP é simples e de fácil implementação, porém não possui segurança. As 
% credenciais do usuário podem ser facilmente decodificadas, visto que a codificação base64 é 
% facilmente reversível, podendo ser realizada em poucos segundos. Também é possível realizar ataques 
% de reprodução, visto que terceiros podem capturar pacotes e replicá-los, mesmo que codificados, 
% podendo obter acesso ao sistema. Este tipo de autenticação não possui proteção contra \emph{proxies} 
% ou \emph{middlewares}, que podem facilmente modificar o corpo da mensagem, e também são vulneráveis 
% a servidores falsificados, que se passam por outros para realizar o roubo de credenciais 
% \cite{GOURLEY2002}. Além destes pontos negativos a autenticação básica é \emph{stateless} (sem 
% estado), cada solicitação é tratada separadamente, sem conexão contínua entre elas. Isso faz com 
% que, enquanto possuir os dados, o cliente continuará mandando os cabeçalhos de autenticação em 
% todas as requisições HTTP subsequentes para o mesmo domínio.

%%%%%% Digest
% Apesar da grande melhora de segurança em relação a autenticação básica, este método possui 
% diversos riscos de segurança. Os cabeçalhos \texttt{WWW-Authenticate} e \texttt{Authorization} 
% possuem certo nível de proteção a manipulação, porém os outros não. Ataques de repetição poderão 
% ser realizados se a implementação de identificadores únicos por desafio nao for realizada. Caso não
% seja estabelecida nenhuma política de força de senha, poderão ser realizados ataques de dicionário, 
% tentando adivinhar a senha e outros parâmetros, visto que o nome do usuário é obtido sem esforço.
% Se a requisição passar por \emph{proxies} hostis ou comprometidos, o cliente pode ficar vulnerável 
% a ataques \emph{man-in-the-middle} \cite{GOURLEY2002}. Este método também é \emph{stateless}, possuindo os mesmo problema citado na autenticação básica.

%%%%%% Session
% A grande vantagem deste método de autenticação é a diminuição do envio das credenciais do usuário nas requisições, diminuindo a janela de ataques, já que os \emph{cookies} são utilizados para a validação das requisições. Por outro lado, os cookies podem ser lidos por outros aplicativos, tornando o sistema exposto a ataques \emph{Cross Site Scripting} (XSS) e \emph{Cross Site Request Forgery} (CSRF). Para evitar ataques XSS, pode-se definir no \emph{cookie} a \emph{flag} \texttt{http-only}, que faz com que o acesso por APIs do lado cliente seja negado \cite{PAPATHANASAKI2022}.