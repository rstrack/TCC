\section{Resultados e Discussão}

%%%%% Basic
A autenticação básica HTTP é simples e de fácil implementação, porém não possui segurança. As 
credenciais do usuário podem ser facilmente decodificadas, visto que a codificação base64 é 
facilmente reversível, podendo ser realizada em poucos segundos. Também é possível realizar ataques 
de reprodução, visto que terceiros podem capturar pacotes e replicá-los, mesmo que codificados, 
podendo obter acesso ao sistema. Este tipo de autenticação não possui proteção contra \emph{proxies} 
ou \emph{middlewares}, que podem facilmente modificar o corpo da mensagem, e também são vulneráveis 
a servidores falsificados, que se passam por outros para realizar o roubo de credenciais 
\cite{GOURLEY2002}. Além destes pontos negativos a autenticação básica é \emph{stateless} (sem 
estado), cada solicitação é tratada separadamente, sem conexão contínua entre elas. Isso faz com 
que, enquanto possuir os dados, o cliente continuará mandando os cabeçalhos de autenticação em 
todas as requisições HTTP subsequentes para o mesmo domínio.

%%%%% Digest
Apesar da grande melhora de segurança em relação a autenticação básica, a autenticação 
\emph{Digest}, este método possui diversos riscos de segurança. Os cabeçalhos 
\texttt{WWW-Authenticate} e \texttt{Authorization} possuem certo nível de proteção a manipulação, 
porém os outros não. Ataques de repetição poderão ser realizados se a implementação de 
identificadores únicos por desafio nao for realizada. Caso não seja estabelecida nenhuma política de 
força de senha, poderão ser realizados ataques de dicionário, tentando adivinhar a senha e outros 
parâmetros, visto que o nome do usuário é obtido sem esforço. Se a requisição passar por 
\emph{proxies} hostis ou comprometidos, o cliente pode ficar vulnerável a ataques 
\emph{man-in-the-middle} \cite{GOURLEY2002}. Este método também é \emph{stateless}, possuindo o 
mesmo problema citado na autenticação básica.

%%%%% Session
Em relação à autenticação baseada em sessão, a grande vantagem é a diminuição do envio das 
credenciais do usuário nas requisições, diminuindo a janela de ataques, já que os \emph{cookies} 
são utilizados para a validação das requisições. Por outro lado, os cookies podem ser lidos por 
outros aplicativos, tornando o sistema exposto a ataques \emph{Cross Site Scripting} (XSS) e 
\emph{Cross Site Request Forgery} (CSRF). Para evitar ataques XSS, pode-se definir no \emph{cookie} 
a \emph{flag} \texttt{http-only}, que faz com que o acesso por APIs do lado cliente seja negado 
\cite{PAPATHANASAKI2022}.

%%%%% OAuth
O protocolo OAuth quando foi publicado, em 2007, tornou-se rapidamente o padrão na indústria para 
delegação de acesso na web. Porém, teve problemas no domínio empresarial, devido ao seu desempenho. 
A comunidade percebeu que o protocolo não era escalável: exige gerenciamento de estado em diferentes 
etapas, gerenciamento de credenciais temporárias e não fornece isolamento do servidor de autorização 
do próprio servidor de recursos protegidos \cite{NOUREDDINE2011}. O OAuth 2.0 resolveu estes 
problemas, facilitando o fluxo ao substituir as assinaturas por \emph{bearer tokens}, utilizando TLS 
durante todo o fluxo, não somente no \emph{handshake} inicial e definindo o servidor de autorização 
separadamente do servidor de recurso, que traz maior flexibilidade \cite{SIRIWARDENA2014}.

Em relação a ataques, ambos são suscetíveis a CSRF. A versão 1.0 é mais suscetível devido a 
falta de uso de TLS em todo seu fluxo, enquanto a versão 2.0 é suscetível a este ataque caso as 
diretrizes de implementação não forem seguidas \cite{FETT2016}. Também acaba se tornando suscetível 
a ataques de \emph{pishing} e \emph{spoofing}, caso não forem seguidas as diretrizes \cite{RFC6819}

\begin{table}[ht]
\centering
\caption{Tabela comparativa entre aspectos de ferramentas de autenticação e autorização}
\label{tab:comparativeTable}
\begin{tabular}{|l|l|l|l|l|}
% Cabeçalho
\hline
Mecanismo
% & Usabilidade
& \begin{tabular}[c]{@{}l@{}}Nível de \\ Segurança\end{tabular}
& Vulnerabilidades
% & \begin{tabular}[c]{@{}l@{}}Facilidade de \\ Implementação\end{tabular}
& Obsolescência
\\ \hline

% Basic Auth
\begin{tabular}[c]{@{}l@{}}HTTP Basic \\ Authentication\end{tabular}
% & Baixa
& \begin{tabular}[c]{@{}l@{}}Extremamente \\ baixo\end{tabular}   
& \begin{tabular}[c]{@{}l@{}}Força-Bruta\\Interceptação\\Reprodução\\\emph{Man-in-the-middle}\\\emph{Spoofing}\end{tabular}
% &
& \begin{tabular}[c]{@{}l@{}}Não\\recomendado\\para sistemas\\modernos\\\end{tabular}
\\ \hline

% Digest Auth
\begin{tabular}[c]{@{}l@{}}HTTP Digest \\ Authentication\end{tabular}
% & Baixa
& Baixo
& \begin{tabular}[c]{@{}l@{}}Força-Bruta\\Reprodução\\\emph{Man-in-the-middle}\end{tabular}
% &
& \begin{tabular}[c]{@{}l@{}}Não\\recomendado\\para sistemas\\modernos\\\end{tabular}
\\ \hline

% Session-Based Auth
\begin{tabular}[c]{@{}l@{}}Session-Based \\ Authentication\end{tabular}
% & Alta
& Médio
& \begin{tabular}[c]{@{}l@{}}CSRF\\XSS (sem HTTPS)\\\end{tabular}
% &
& \begin{tabular}[c]{@{}l@{}}Amplamente\\utilizado\\mas possui\\opções\\mais modernas\\disponíveis\\\end{tabular}
\\ \hline

% Token-Based Auth
\begin{tabular}[c]{@{}l@{}}Token-Based \\ Authentication\end{tabular}
% & Alta
& 
& 
% &
& \begin{tabular}[c]{@{}l@{}}Amplamente\\utilizado\\mas possui\\opções\\mais modernas\\disponíveis\\\end{tabular}
\\ \hline

% OAuth
OAuth
% & Alta
& Alto
& \begin{tabular}[c]{@{}l@{}}CSRF\\\emph{Spoofing}\\\end{tabular}
% &
& \begin{tabular}[c]{@{}l@{}}Obsoleto em\\relação ao\\OAuth 2.0\\porém ainda\\aplicável\\\end{tabular}
\\ \hline

% OAuth 2.0
OAuth 2.0
% & Alta
& Alto
& \begin{tabular}[c]{@{}l@{}}\emph{Pishing}\\\end{tabular}
% &
& \begin{tabular}[c]{@{}l@{}}Amplamente\\utilizado e\\recomendado\\\end{tabular}
\\ \hline

% OpenID Connect
OpenID Connect
% & Alta
& Alto
& 
% &
& \begin{tabular}[c]{@{}l@{}}Amplamente\\utilizado e\\recomendado\\\end{tabular}
\\ \hline
\end{tabular}
\end{table}