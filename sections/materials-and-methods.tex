\section{Materiais e Métodos}

Para o desenvolvimento do presente estudo, utilizou-se uma abordagem qualitativa. Os dados coletados relativos às ferramentas de autenticação e autorização são descritivos em sua totalidade.

Realizou-se uma extensa pesquisa de dados relacionada aos métodos de autenticação HTTP \emph{Basic Autentication}, HTTP \emph{Digest Authentication}, \emph{Session-Based Authentication}, \emph{Token-Based Authentication}, OAuth, OAuth 2.0 e OpenID. Para a coleta de dados, foram utilizados artigos científicos publicados em revistas e conferências internacionais, livros e RFCs (\emph{Research for Comments}), documentos técnicos mantidos pela IETF (\emph{Internet Enginnering Task Force}).