\subsection{Autenticação Baseada em Sessão}

Uma sessão web é uma troca de informações semipermanente entre um cliente e um servidor web \cite{CALZAVARA2017}. O mecanismo de gerenciamento de estados para o HTTP, baseado em sessões, foi especificado na RFC 2109 \cite{RFC2109} com sua versão mais atual especificada na RFC 6265 \cite{RFC6265}. Este mecanismo utiliza o termo \emph{cookie} para se referir às informações de estados que são passadas entre o servidor e o cliente, e salvas no cliente \cite{RFC2109}. 

A autenticação baseada em sessão é o método mais comum de autenticação em sistemas web. Neste método, após o envio de credenciais de acesso e validação do usuário por meio de uma requisição HTTP, o servidor gera um \emph{cookie}, armazena-o e envia-o pelo cabeçalho \texttt{Set-Cookie} da resposta para o cliente. O cliente salva o valor, que é enviado no cabeçalho \texttt{Cookie} em toda requisição para o mesmo servidor de origem \cite{PAPATHANASAKI2022}. Um exemplo do funcionamento deste método é mostrado na figura \ref{fig:sessionAuth}

\begin{figure}[ht]
  \centering
  \includegraphics[width=.8\textwidth]{Session-Based Authentication.png}
  \caption{Exemplo de Session-Based Authentication}
  \label{fig:sessionAuth}
\end{figure}

A grande vantagem deste método de autenticação é a diminuição do envio das credenciais do usuário nas requisições, diminuindo a janela de ataques, já que os \emph{cookies} são utilizados para a validação das requisições. Por outro lado, os cookies podem ser lidos por outros aplicativos, tornando o sistema exposto a ataques \emph{Cross Site Scripting} (XSS) e \emph{Cross Site Request Forgery} (CSRF). Para evitar ataques XSS, pode-se definir no \emph{cookie} a \emph{flag} \texttt{http-only}, que faz com que o acesso por APIs do lado cliente seja negado \cite{PAPATHANASAKI2022}.