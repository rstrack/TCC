% \section{Conclusão}
\section{Considerações Finais}

Este trabalho teve como objetivo realizar um estudo comparativo dos diferentes mecanismos de 
autenticação e autorização em sistemas web, visando fornecer uma análise aprofundada que auxilie na 
escolha adequada desses mecanismos em projetos. O estudo buscou oferecer uma compreensão ampla das 
características, pontos fortes e limitações de cada mecanismo, a fim de permitir a seleção correta 
e a implementação eficiente das medidas de segurança necessárias.

Ao longo do trabalho, foram apresentados mecanismos de autenticação e autorização, dentre eles a 
autenticação básica HTTP, autenticação \emph{digest}, autenticação baseada em sessão, OAuth e OAuth 2.0.
Foram apresentadas algumas características relacionadas a vulnerabilidades a ataques e 
obsolescência destes mecanismos. Os próximos passos deste estudo são realizar a abordagem dos 
métodos de autenticação baseada em \emph{token} e OpenID, além de levantar os dados restantes para 
a elaboração completa da tabela comparativa entre os mecanismos, sobre as outras características 
definidas anteriormente.

A segurança dos sistemas web é um aspecto crucial que não pode ser negligenciado. A escolha e 
implementação correta dos mecanismos de autenticação e autorização são fundamentais para garantir 
a proteção dos dados e recursos dos usuários. Por meio deste estudo comparativo, espera-se que os 
desenvolvedores e profissionais da área possam tomar decisões mais informadas ao selecionar os 
mecanismos adequados de acordo com as necessidades de cada projeto, contribuindo assim para a 
construção de sistemas web mais seguros e confiáveis.