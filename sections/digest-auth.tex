\subsection{Autenticação Digest}

A autenticação \emph{Digest} foi especificada na RFC 2019 \cite{RFC2019}, porém também foi 
realocada para a RFC 2617. Foi desenvolvida para ser uma alternativa mais compatível e segura  para 
a autenticação básica, corrigindo as falhas mais graves da mesma, como a falta de criptografia de 
senhas, vulnerabilidade a captura e reprodução de pacotes e proteção contra vários outros tipos comuns de ataques \cite{GOURLEY2002}.

Assim como a autenticação básica HTTP, a \emph{Digest} é baseada no paradigma 
desafio-resposta \cite{RFC7616}. A diferença é que foram adicionados diversos parâmetros nos cabeçalhos, para identificação única de desafios, nível de qualidade de proteção, especificação de algoritmo de hashing utilizado entre outros recursos \cite{CHAPMAN2012}. Por padrão, o algoritmo utilizado é o MD5, porém na RFC 7616 foram adicionados e recomendados os algoritmos SHA-256 e SHA-512/256 \cite{RFC7616}.

O parâmetro \texttt{response} é a principal parte do cabeçalho \texttt{Authorization}: ele contém 
uma concatenação criptografada de dados da requisição, como nome do usuário, \texttt{realm}, senha, 
método HTTP, URL, entre outros parâmetros, todos separados por dois pontos. \cite{CHAPMAN2012}. O cliente realiza o cálculo resultante no valor de \texttt{response}. O servidor também o faz e 
compara com o valor recebido. Caso as credenciais sejam válidas, retorna-se o status HTTP 200 (OK)
e o cabeçalho \texttt{Authentication-Info}, que contém parâmetros utilizados para uma futura 
autenticação, autenticação mútua e reenvio de parâmetros para confirmação de legitimidade. Um 
exemplo do funcionamento deste método de autenticação é mostrado na Figura \ref{fig:digestAuth}.

\begin{figure}[ht]
  \centering
  \includegraphics[width=.8\textwidth]{Digest Authentication (Simplified).png}
  \caption{Exemplo de autenticação Digest}
  \label{fig:digestAuth}
\end{figure}

Apesar da grande melhora de segurança em relação a autenticação básica, este método possui 
diversos riscos de segurança. Os cabeçalhos \texttt{WWW-Authenticate} e \texttt{Authorization} 
possuem certo nível de proteção a manipulação, porém os outros não. Ataques de repetição poderão 
ser realizados se a implementação de identificadores únicos por desafio nao for realizada. Caso não
seja estabelecida nenhuma política de força de senha, poderão ser realizados ataques de dicionário, 
tentando adivinhar a senha e outros parâmetros, visto que o nome do usuário é obtido sem esforço.
Se a requisição passar por \emph{proxies} hostis ou comprometidos, o cliente pode ficar vulnerável 
a ataques \emph{man-in-the-middle} \cite{GOURLEY2002}. Este método também é \emph{stateless}, possuindo os mesmo problema citado na autenticação básica.